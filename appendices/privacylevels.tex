\section{Privacy level mapping}
\label{appendix:privacy-levels}
In our example data transformations, four levels of increasing privacy are proposed, based partially on the \gls{GDPR} definition of sensitive personal data and on the types of identifiers defined in section \ref{sec:data-deid}. However, it is important to note that this is only a practical choice and forms a trade-off between granularity and necessary work (for setting up all the transformations required for a certain level). Privacy experts may have more valuable opinions on the correct number of privacy levels and how these should be communicated to a user. Future research could help to find a rigorous definition for these privacy levels, for example based on some leakage metric. This is elaborated in FW \ref{fw:privacy-levels}.\\

\noindent \textbf{Level 1: all data} No data transformations are applied, all data is passed on to the requesting application.\\

\noindent \textbf{Level 2: Removal of sensitive personal data} Sensitive personal data, as defined by the GDPR \citep{GDPR}, is removed from the dataset. This includes data consisting of racial or ethnic origin, political opinions, religious or philosophical beliefs, etc. Thus, the tactic \textit{Remove} is applied to all data elements matching this definition.\\

\noindent \textbf{Level 3: Pseudonymization/generalisation of direct identifiers} Since level 3 is a stronger version of level 2, sensitive personal data is removed first. Additionally, direct personal identifiers are pseudonymized or generalized/perturbed. Concretely, the tactics \textit{Pseudonym}, \textit{Encrypt}, \textit{Perturbation}, ... may be applied here, depending on the specific data attribute. Examples are the replacement of names by placeholders, the perturbation or adding a placeholder for birth dates (such that the exact date is obscured, but the age is still correct), the removal of street names and numbers while keeping larger geographic areas such as cities, etc. This makes the data still relatively accurate, while direct identification of the user is made impossible.\\

\noindent \textbf{Level 4: Pseudonymization/generalisation of (in)direct identifiers} In addition to direct identifiers, also indirect identifiers are now modified or removed. The same \gls{PETs} and tactics are used, but are now applied more strictly and to more data attributes. For example, when perturbing birth dates, now the exact age is not kept exactly, but it is changed to a range within the exact age. Cities may also be perturbed when possible, but keeping for example the province or state. Other indirect identifiers such as genders may also be modified or removed.
