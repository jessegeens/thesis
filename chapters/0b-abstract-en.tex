\begin{abstract}
  In recent years, the internet has become evermore centralized \citep{internet-report}. This has many negative side effects, such as decreased competition and a lack of access to personal data \citep{big-tech-innovation, platform-monopolies}. Solid \citep{solid} is a new draft W3C specification that aims to combat this by introducing pods. Pods are decentralized datastores where user data is kept for many different applications. In this manner, the user retains control over his personal data, and the same data can be used across different services.
  
  These pods provide a number of possibilities, but also come with technological challenges. One such challenge is secure and protected data aggregation across different pods. Such data aggregations impose privacy risks, as well as scalability issues. This dissertation presents a server-level middleware that aims to partially resolve these issues by presenting two novel technologies: privacy filters, and a new token mechanism that supports decentralized delegation.
  
  Privacy filters is a technology that enables more granularity in the privacy of shared resources. When a request is made, the middleware selects a number of transformations to perform on the requested resource, based on the user's settings and contextual parameters such as the requesting application and the resource's data type. To subsequently attain a token mechanism that supports decentralized delegation, this dissertation investigates the use of macaroons as a token mechanism in Solid. Macaroons not only provide decentralized token delegation, but are also much more efficient to generate and verify. Moreover, they allow third-party attestations, which is a useful property for realizing pods belonging to a group of agents.
  
  The middleware presented in this dissertation has been evaluated on the basis of three use cases. For privacy filters, a number of performance experiments have demonstrated that the overhead of the middleware is tolerable for smaller resources. Rewriting resources of around 300KB using three transformations causes an overhead of around 50\% in the request duration. However, larger resources (around 3MB) have overheads that become a five-fold of the original request duration. As such, a more efficient method for resource rewriting may be necessary here, although this issue can be lessened by making use of caching and precomputation. Performance experiments on the generation and verification of macaroons have illustrated that macaroons offer large performance benefits. The throughput of generating and verifying macaroons is seven and eleven times higher respectively, compared to \acrshort{DPoP} using the ES256 algorithm. Finally, there are also theoretical gains for decentralized delegation in a reduced number of needed interactions for realizing token delegation.
\end{abstract}