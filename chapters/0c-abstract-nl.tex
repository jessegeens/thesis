\begin{abstract*}
  \begin{otherlanguage}{dutch}
  De afgelopen jaren is het internet steeds meer gecentraliseerd geworden \citep{internet-report}. Dit heeft vele negatieve gevolgen, waaronder verminderde concurrentie en een gebrek aan toegang tot persoonlijke gegevens \citep{big-tech-innovation, platform-monopolies}. Solid \citep{solid} is een nieuwe, voorlopige W3C specificatie die dit probleem tracht op te lossen door het introduceren van datakluizen. Datakluizen zijn gedecentraliseerde opslagplaatsen voor data waar gegevens voor verschillende toepassingen worden bijgehouden. Op deze manier bewaart de gebruiker controle over zijn persoonlijke gegevens en kan dezelfde data door verschillende diensten gebruikt worden. Deze datakluizen brengen een hoop mogelijkheden met zich mee, maar evenzeer technologische uitdagingen. 
  
  E{\'e}n van zo'n uitdagingen is het mogelijk maken van veilige en beschermde gegevensaggregatie over verschillende datakluizen heen. Dergelijke aggregaties brengen privacyrisico's met zich mee, net als schaalbaarheidsproblemen. Typisch vraagt zo'n aggregator volledige bronnen op om vervolgens de resulterende aggregatie te anonimiseren, wat privacy-onvriendelijk is. Bovendien moeten dergelijke aggregators mogelijk gegevens van verscheidene diensten combineren, hetgeen autorisatie-uitdagingen met zich meebrengt. Deze thesis introduceert een middleware op het niveau van de server, die tracht deze problemen gedeeltelijk op te lossen dankzij het introduceren van twee nieuwe technologie{\"e}n. Deze technologie{\"e}n zijn privacy filters en een nieuw mechanisme voor toegangstokens dat gedecentraliseerde tokenafvaardiging ondersteunt.

  Privacy filters is een technologie die bronnen herschrijft bij een verzoek, gebaseerd op de gebruiker zijn privacy-instellingen en contextuele parameters. Deze parameters omvatten welke toepassing het verzoek verzonden heeft en wat het datatype is van de verzochte bron. Dit laat toe om meer granulariteit te bereiken in de privacy van gedeelde bronnen. Om vervolgens een tokenmechanisme te bekomen dat gedecentraliseerde tokenafvaardiging ondersteunt, onderzoekt deze thesis het gebruik van macaroons als nieuw tokenmechanisme binnen Solid. Macaroons ondersteunen niet alleen gedecentraliseerde tokenafvaardiging, maar zijn ook effici{\"e}nter om te genereren en verifi{\"e}ren. Bovendien laten ze attestaties van derde partijen toe (\textit{third-party attestations}), wat een nuttige eigenschap is om datakluizen te bekomen die gedeeld worden door een groep gebruikers.
  
  De middleware die deze thesis voorstelt werd ge{\"e}valueerd op grond van drie gebruiksscenario's. Voor privacy filters hebben prestatie-experimenten aangetoond dat de overhead van de middleware toelaatbaar is voor kleinere bronnen. Het herschrijven van bronnen tot 300KB, gebruikmakende van drie transformaties, leidt tot een overhead van ongeveer 50\%. Grotere bronnen (rond de 3MB) hebben echter een overhead die een vijfvoud wordt van de oorspronkelijke duur van het verzoek, maar dit probleem kan verminderd worden door gebruik te maken van caches of met behulp van voorberekeningen. Prestatie-experimenten die werden uitgevoerd op het genereren en verifi{\"e}ren van macaroons hebben aangetoond dat hier sterke prestatievoordelen aan vasthangen. De doorvoer voor het genereren en verifi{\"e}ren van macaroons is respectievelijk zeven en elf maal groter dan het \acrshort{DPoP} systeem wanneer dit gebruik maakt van het ES256 algoritme. Ten slotte zijn er ook theoretische winsten voor gedecentraliseerde tokenafvaardiging, gezien dit mechanisme een verminderde interactiekost nodig heeft voor het afvaardigen van een toegangstoken in vergelijking met een gecentraliseerd mechanisme zoals \acrshort{DPoP}.
  \end{otherlanguage}
\end{abstract*}