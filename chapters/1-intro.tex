\chapter{Introduction}
\label{cha:intro}
%The first contains a general introduction to the work. The goals are
%defined and the modus operandi is explained.

\section{Problem Statement}
In recent years, the use of public cloud providers has seen an enormous surge. However, this comes with necessary risks, especially for sectors where the stored data is highly sensitive (e.g. patient data). Combined with recently strengthened data protection regulations, such as GDPR, software providers may be wary of using third-party cloud infrastructure. To tackle this problem, many data protection techniques and middleware have been proposed. 

Middleware solutions in general, and in particular data access middleware, aim to facilitate software systems and developers to employ complex concepts. These solutions often operate in a distributed fashion. A prominent approach to achieve this goal is to use abstractions. For example, recent research [4,3] presents distributed data-access middleware solutions by introducing several abstraction models to enable software developers to perform search and computation over encrypted data transparently. In general, using different data protection tactics, especially those that rely on cryptography, is not a trivial task for enterprise software developers. It is not easy for a non-security expert to understand the complex threat models and implement the protocols and primitives correctly. This is an error-prone process.

In tomorrow's business landscape, we need to use advanced data protection tactics. For example, a concrete application case could be multiple hospitals that want to compute the average Body Mass Index (BMI) of their patients, without revealing any sensitive patient information to another hospital or a third party. Next to this case, a medical research institute requires to combine two data sets that each belong to a different hospital to calculate certain statistics. However, no hospital is allowed to share their patients' data with any other entity, be it another hospital or a research institute.

Many techniques exist for privacy-aware computation and secure data aggregation such as trusted computing, secure multi-party computation (MPC), homomorphic encryption, and more. 


\section{Contributions}
\lipsum[6-7]

\section{Thesis Outline}
\lipsum[2-3]