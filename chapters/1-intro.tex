\chapter{Introduction}
\label{cha:intro}
%The first contains a general introduction to the work. The goals are
%defined and the modus operandi is explained.

\section{Problem Statement}

\section{Motivation}
The motivation for writing this thesis consists of a number of factors. 

On a personal note, I am very grateful to be able to work on the topic of Solid and privacy-enhancing technologies. I believe that decentralizing the web is a very noble goal and I am very glad to be able to contribute to this. I am also very interested in technologies related to privacy, and I am sure that completing this thesis will introduce me in-depth to a number of very important concepts in the field.

From a societal point-of-view, this thesis also has a two-fold motivation. First of all, Solid is a very new technology that is receiving a lot of attention and investments over the last few months, both in the public and private sector. There are deals with the Flemish \citepjournal{solid-flanders} and Swedish \citepjournal{solid-sweden} governments, and Inrupt (MIT's Solid spin-off company) has just received a thirty million dollar Series A investment \citepjournal{inrupt-seriesa}. This will lead to an influx of developers and users into the ecosystem, which will result in an increase of untrusted Solid applications. 
On the other hand, especially since the release of the GDPR and some data privacy scandals, there is a big focus on data privacy recently, both from companies and users \citepjournal{mckinsey-privacy}. This research combines both of these reasons to realise a potentially big impact by improving the data privacy capabilities of the Solid ecosystem. 


\section{Contributions}

\section{Thesis Outline}