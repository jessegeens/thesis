\chapter{Introduction}
\label{cha:intro}
%The first contains a general introduction to the work. The goals are
%defined and the modus operandi is explained.

This dissertation presents a middleware for secure and protected data aggregation. Section \ref{sec:motivation} discusses the importance of such a middleware. Subsequently, a concrete problem statement is discussed in section \ref{sec:problem}. Afterwards, section \ref{sec:contributions} lists the contributions this dissertation makes. To conclude, section \ref{sec:outline} gives an overview of the structure of the remaining parts of the text.

\section{Motivation}
\label{sec:motivation}
In recent years, the internet has become more and more centralized \citep{internet-report}. This has many negative side effects: powerful companies have control over vast amounts of data, providing new market insights or personalization options, which in turn lead to more market power \citep{big-tech-innovation, platform-monopolies}. Lacking data on such large scales, smaller enterprises do not have access to these capabilities. This stifles competition as it makes it harder for these smaller companies to find a position in the market, limiting innovation. 

Furthermore, the lack of access to personal data has also manifested itself at the user's side. It is often hard or impossible to share data between competing platforms, and users lack insights into what data is held by what companies. The introduction of the \gls{GDPR} \citep{GDPR} aimed to combat this by giving users the right to look at their personal data, but this still requires a lot of effort from the user as this data is splintered across many different services and platforms. This contrasts with an ever-increasing focus on data privacy from users \citepjournal{mckinsey-privacy}.

To solve these data ownership problems, the Solid \citep{solid} specification was introduced. Solid is a very new technology that is receiving a lot of attention and large investments over the last few months. These investments come from multiple governments \citepjournal{solid-flanders, solid-sweden} as well as industry \citepjournal{inrupt-seriesa}. This will hopefully lead to an influx of developers and users into the ecosystem, which provides many opportunities from a data perspective. These data opportunities offer the most value when combined across pods, which highlights the need for secure data aggregation mechanisms.

\section{Problem statement: data aggregation in the decentralized web}
\label{sec:problem}
 Solid is a web specification with the aim of decentralizing the web again. The specification introduces a number of concepts to achieve this. Every user (a person, automated agent, group, ...) is given his own WebID: an identity on the web. Furthermore, WebIDs often also have one or more coupled \textit{pods}. Pods are personal online data vaults, and form a standardized way to store a user's data. This allows the user to re-use the same data across different services, and to fine-tune which applications have access to which data.

Solid is a very recent specification, and as such there remain many open challenges. One of these challenges is secure aggregation of data across pods. Pods often hold very interesting information that, when combined with information from other pods, can lead to very interesting insights for users, companies and other stakeholders. However, it is technically very challenging to aggregate such data from different pods. Many aspects come in to play here. Firstly, data aggregation across Solid pods can lead to exposure of sensitive data. Hence, a mechanism to release data in a more fine-grained and controlled manner is necessary. Secondly, aggregating data across many pods implies that many authentication requests will have to be made. As such, a scalable, efficient and flexible authentication and authorization mechanism must be devised. Preferably, such a mechanism would also work in a decentralized manner, since complex aggregations can depend on many services.

The aim of this dissertation is thus to come up with possible solutions for secure and protected data aggregation in the Solid ecosystem. 
To concisely summarize this section, we formulate the following high-level research question.\\

\noindent \textbf{RQ: Can we design a scalable middleware solution aimed at facilitating data aggregation in a secure and protected manner in a setting where data is distributed across various parties?
}

\section{Contributions}
\label{sec:contributions}
This dissertation presents a Middleware for Aggregation Security in Solid (\middleware{}), a conceptual server-side middleware for facilitating secure and protected data aggregation in Solid. Concretely, this dissertation makes the following contributions to the research domain of Solid:
\begin{enumerate}
    \item An overview of literature in the domain of \acrlong{PETs}, accompanied by a discussion of each technology and its usability within the context of Solid.
    \newpage
    \item Privacy filters, a novel, customizable and seamless approach to limit the information leakage of data exposed through Solid Pods. This mechanism is implemented in a prototype extension of the \acrlong{CSS}\footnote{\url{https://github.com/CommunitySolidServer/CommunitySolidServer}} and evaluated.
    \item A mechanism for realizing decentralized delegation of access tokens in Solid, which also enables more efficient generation of tokens and enables tokens with third-party attestations. This mechanism is also accompanied by a prototype and benchmarked.
\end{enumerate}

\section{Dissertation Outline}
\label{sec:outline}
This dissertation aims to develop possible solutions for realizing a middleware for secure and protected data aggregation in Solid. The next sections of this chapter discuss a number of use cases which illustrate the problem statement and its complexities.

To better understand the context in which this dissertation is set, chapter \ref{cha:background} introduces some necessary background information. This chapter covers all the topics on which this dissertation is built, such as an in-depth explanation of the Solid protocol, as well as an introduction to macaroons, a novel type of access token. 

Chapter \ref{cha:solution-overview} starts by introducing a number of use cases which highlight the need for an aggregation middleware. Furthermore, a generic architecture for data aggregation in Solid is introduced. This architecture aids in researching what the concrete bottlenecks are and is thus used to develop some concrete goals for the middleware. Based on the presented use cases and goals, a number of requirements and a relevant attacker model are proposed. The final section of this chapter then gives a system overview of \middleware{}, explaining on a high level the two main components of the middleware.

The first building block necessary for such a middleware system concerns the privacy aspect. Therefore, chapter \ref{cha:analysis} studies a number of widely used \gls{PETs} and discusses their relevance in the context of Solid. Some of these \gls{PETs} will be used in the developed solution, while devising solutions using the remaining applicable \gls{PETs} is considered future work (discussed in chapter \ref{cha:conclusion}).

Chapter \ref{cha:privacy-filters} subsequently introduces privacy filters, one of the components of the solution, in detail. This component focuses on rewriting requests to render data more private when it is requested by an aggregator or application. The chapter also discusses the concrete implementation of this component. The main findings of section \ref{cha:analysis} will be used to develop this solution.

Having developed a possible solution to the privacy problem, the next challenge is \textit{(decentralized) delegation}. This is discussed in chapter \ref{cha:macaroons-solid}. Aggregators can be quite complex, consisting of multiple workers, and they need long-lived access to the data in a safe manner. Macaroons are studied as a possible remedy, and their advantages and drawbacks are discussed. Combining these technologies, the chapter concludes with a possible architecture in which privacy filters and macaroons are used to build a secure and protected data aggregator.

To better understand the applicability of this dissertation's contributions in terms of performance and feasibility chapter \ref{cha:evaluation} validates and evaluates the results. This is done by discussing theoretical shortcomings, as well as performing some experiments to measure the performance of developed prototypes. 

Finally, chapter \ref{cha:conclusion} summarizes the main findings of this dissertation. The dissertation is then concluded by discussing potential areas of future work, which may hopefully inspire the reader to perform further research in this domain.
