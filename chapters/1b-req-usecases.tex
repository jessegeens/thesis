\chapter{Use cases and requirements}
This chapter introduces a number of use cases that illustrate the problem statement. Based on these use cases, a number of requirements are derived. Furthermore, the adversary model that is considered in this thesis is introduced.


\section{Use Cases: Health and Finance}
\label{sec:usecases}
To illustrate the problem statement, this section describes a number of use cases which highlight the need for a middleware for secure data aggregation in Solid. These use cases illustrate the complexities and challenges that come with developing such a middleware, and will serve as a guide throughout the development of a solution.

\subsection{Exercise data}
\label{usecase:ex-data}
In this use case, exercise data from an application such as Strava is stored on a Solid pod by a user. She wishes to export this data to a ranking board application to see which of her friends runs the fastest. However, she does not wish to share her exact heart rate since this is sensitive data and might leak information about her fitness. This data is stored in the TCX file format\footnote{Schema: \url{https://www8.garmin.com/xmlschemas/TrainingCenterDatabasev2.xsd}}. An advantage of this format is that it is supported by popular tools such as Strava and can be imported/exported by exercise trackers such as Garmin devices.

\subsection{Personal finance}
\label{usecase:personal-finance}
This use case describes a scenario wherein a user has stored all his transactions on his Solid pod. He wishes to see some trends and statistics about his spending, and compare this to similar households. An example of such a statistic is "how much do I spend on groceries every week?".  However, expenses are very sensitive data, especially when these are exact numbers and store locations. Therefore, the middleware should filter out the most sensitive information yet still receive relevant statistics. This can be done, for example, by removing direct identifiers and perturbing exact spending at individual transactions. For example, for entries of the type "Bob spent \texteuro 5,82 at Colruyt Leuven on 8/11/2021 16:53", the user wishes that this is modified into something similar to "User87532 spent \texteuro 5 at Supermarket on 8/11/2021". This way, trends in the spending are kept (by perturbing exact amounts to nearby integers, and replacing exact stores with store types). Information such as "you spent \texteuro 400 in supermarkets this month" will still be available (and relatively accurate), without giving away exact details. For this use case a custom data format is used, since most standard data schemes for financial information are overly complex for a proof-of-concept.

\subsection{Aggregated view on personal health data streams}
While the previous two use cases illustrate potential problems which call for secure aggregation methods, they lack external validation. The third use case is therefore taken from one of the SolidLab\footnote{A research project initiated by the Flemish government \citepjournal{solid-flanders}} challenges. Concretely, the challenge \textit{aggregated view on sensitive personal health data streams} is studied\footnote{See \url{https://github.com/SolidLabResearch/Challenges/issues/16}}.

This use case describes a scenario wherein a caretaker wishes to gain insights on all her patients, without knowing exact patient details. These insights are provided through a dashboard, which shows some key statistics about her patient population. Examples of such statistics are average heart rates, average number of steps taken throughout the day, what percentage of the population woke up before 9 am, etc. Of course, the collection and aggregation of such data does not come without issues. Data is derived from activity trackers and IoT sensors, which regularly update data in the user's pod. A secure aggregator must then collect data from all these pods, combine these into aggregate statistics, and write this to a new pod accessible to the caretaker.


