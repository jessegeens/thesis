\chapter{Background}
\label{cha:background}
The proposed middleware builds heavily on the Solid specification \citep{solid}: a proposed W3C specification that wishes to decentralize the web. This chapter introduces the philosophy behind solid, and the building blocks the specification consists of.

\section{Overview}
\begin{quote}{\href{https://solidproject.org}{solidproject.org}}
    Solid is a proposed set of conventions and tools for building \textit{decentralized applications} based on Linked Data principles.
\end{quote}

\noindent Solid is a proposed W3C specification that wishes to decentralize the web by giving people control over their data. Solid aims to realize this by giving people an online datastore called a pod. Users can then login to applications using an authentication mechanism provided by Solid, and the application can in turn access data stored in the user's pod.

By storing all the data inside the user's pod instead of on the application server, the user retains complete control over his data. He can flexibly choose what data to share with what applications. Furthermore, storing the data in a pod reduces vendor lock-in: when the user wishes to switch from one service to another, he can simply give the new application access to the data he already possesses. 

In Solid, there are two types of resources: linked and non-linked data resources. Non-linked data resources are the usual kinds of data (binary or text), while linked data follows the 
 RDF\footnote{Resource Description Framework} specifications, by using a content representation such as Turtle \citep{turtle}. These linked-data standards (sometimes also called the \textit{semantic web} define a way to model data using vocabularies or ontologies, yielding a standardized way of saving certain kinds of data (such as movies\footnote{\url{https://schema.org/Movie}}).

\section{Authentication}
Authenticating a user is the act of verifying a user's claimed identity. Identities in solid (both for users and other agents) follow the WebID standard \citep{webid}: URIs act as universal identifiers. An example WebID is:

\begin{center}
   \texttt{https://jessegeens.solidcommunity.net/profile/card\#me}\\
\end{center}

\noindent These WebID URIs can identify several things: users, software agents, or other things (even if they do not exist on the web). The WebID URI references a WebID Profile Document. This document contains information about the agent who is the referent of the WebID URI. However, an important distinction must be made. When the \textit{\texttt{\#me}} is included in the URI (or more in general, the hashtag), this URI refers to an agent. When the hashtag is omitted, the URI refers to the document describing this agent.

While the WebID standard provides identities, it does not yet verify them. The actual authentication (or, verifying that the agent actually controls the WebID he claims to control) in Solid uses the WebID-OIDC protocol. This protocol draws inspiration from OAuth 2 and from OpenID Connect. The below workflow, drawn from the WebID-OIDC repository, explains the basic mechanism of the protocol:

\begin{quote}{\href{https://github.com/solid/webid-oidc-spec}{webid-oidc-spec repository}}

    1. \textbf{Initial Request}: Alice (unauthenticated) makes a request to bob.example, receives a \texttt{HTTP 401 Unauthorized} response, and is presented with a 'Sign In With...' screen.\\
    2. \textbf{Provider Selection}: She selects her WebID service provider by clicking on a logo, typing in a URI (for example, alice.solidtest.space), or entering her email.\\
    3. \textbf{Local Authentication}: Alice gets redirected towards her service provider's own Sign In page, thus requesting https://alice.solidtest.space/signin, and authenticates using her preferred method (password, WebID-TLS certificate, FIDO 2 / WebAuthn device, etc).\\
    4. \textbf{User Consent}: (Optional) She's presented with a user consent screen, along the lines of "Do you wish to sign in to bob.example?".\\
    5. \textbf{Authentication Response}: She then gets redirected back towards https://bob.example/resource1 (the resource she was originally trying to request). The server, bob.example, also receives a signed ID Token from alice.solidtest.space that was returned with the response in point 3, attesting that she has signed in.\\
    6. \textbf{Deriving a WebID URI}: bob.example (the server controlling the resource) validates the ID Token, and extracts Alice's WebID URI from inside it. She is now signed in to bob.example as user https://alice.solidtest.space/\#i.\\
    7. \textbf{WebID Provider Confirmation}: bob.example confirms that solidtest.space is indeed Alice's authorized OIDC provider (by matching the provider URI from the iss claim with Alice's WebID).

\end{quote}

\section{Authorization}
Authorization within the Solid project builds on the Web Access Control standard \citep{wac}. The WAC standard provides a method to define authorization conditions for resources in a Pod using an Access Control List. Every resource is coupled with an ACL resource - either directly, or by inheriting it from a parent resource. These ACL resources are specified in a turtle file.  Authorization conditions consist of three elements: access objects (the resource the ACL resource applies to), access modes (read, write, append or control) and access subjects (the agents performing the request). An example ACL resource, taken from the Solid Community Server \footnote{See \url{https://github.com/solid/community-server}}, is presented below:
\lstinputlisting[style=turtle, title=.acl]{code/acl.ttl}

\section{Linked Data Protocol and Containers}
\lipsum[2-3]