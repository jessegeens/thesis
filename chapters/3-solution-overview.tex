\chapter{Use cases, requirements and system overview}
\label{cha:solution-overview}
This chapter starts by introducing various use cases that illustrate the problem statement. To illustrate the problems that occur in these use cases more clearly, a generic data aggregation architecture is presented. This architecture will highlight a number of problems that currently exist. Based on this architecture and the use cases, a number of requirements are derived, along with an adversary model.

Subsequently, based on all these requirements, a high-level overview of the solution is given. This solution is developed as a prototype, and is named \middleware{}: Middleware for Aggregation Security and protection in Solid. \middleware{} is thus a middleware for performing secure and protected data aggregation on top of the Solid protocol. 

The next chapters then start by studying a number of \gls{PETs} in order to determine which technologies are usable in \middleware{}. The chapters following this finally discuss the components of the solution in detail.

\section{Use cases: health and finance}
\label{sec:usecases}
This section describes three use cases which highlight the need for a middleware for secure data aggregation in Solid. These use cases illustrate the complexities and challenges that come with developing such a middleware, and will serve as a guide throughout the development of a solution. The use cases are located in the domains of health and finance, since these come traditionally with sensitive data. Furthermore, data in these domains can also lead to interesting insights, forming ideal candidates for data aggregation.

\subsection{Exercise data}
\label{usecase:ex-data}
In this use case, exercise data from an application such as Strava is stored on a Solid pod by a user. She wishes to export this data to a ranking board application to see which of her friends runs the fastest. However, she does not wish to share her exact heart rate since this is sensitive data and might leak information about her fitness. This data is stored in the TCX file format\footnote{Schema: \url{https://www8.garmin.com/xmlschemas/TrainingCenterDatabasev2.xsd}}. An advantage of this format is that it is supported by popular tools such as Strava and can be imported/exported by exercise trackers such as Garmin devices.

\subsection{Personal finance}
\label{usecase:personal-finance}
This use case describes a scenario wherein a user has stored all his transactions on his Solid pod. He wishes to see some trends and statistics about his spending, and compare this to similar households. An example of such a statistic is "how much do I spend on groceries every week?".  However, expenses are very sensitive data, especially when these are exact numbers and store locations. Therefore, the middleware should filter out the most sensitive information yet still receive relevant statistics. This can be done, for example, by removing direct identifiers and perturbing exact spending at individual transactions. For example, for entries of the type "Bob spent \texteuro 5,82 at Colruyt Leuven on 8/11/2021 16:53", the user wishes that this is modified into something similar to "User87532 spent \texteuro 5 at Supermarket on 8/11/2021". This way, trends in the spending are kept (by perturbing exact amounts to nearby integers, and replacing exact stores with store types). Information such as "you spent \texteuro 400 in supermarkets this month" will still be available (and relatively accurate), without giving away exact details. For this use case a custom data format is used, since most standard data schemes for financial information are overly complex for a proof-of-concept.

\subsection{Aggregated view on personal health data streams}
While the previous two use cases illustrate potential problems which call for secure aggregation methods, they lack external validation. The third use case is therefore taken from one of the SolidLab\footnote{A research project initiated by the Flemish government \citepjournal{solid-flanders}} challenges. Concretely, the challenge \textit{aggregated view on sensitive personal health data streams} is studied\footnote{See \url{https://github.com/SolidLabResearch/Challenges/issues/16}}.

This use case describes a scenario wherein a caretaker wishes to gain insights on all her patients, without knowing exact patient details. These insights are provided through a dashboard, which shows some key statistics about her patient population. Examples of such statistics are average heart rates, average number of steps taken throughout the day, what percentage of the population woke up before 9 am, etc. Of course, the collection and aggregation of such data does not come without issues. Data is derived from activity trackers and IoT sensors, which regularly update data in the user's pod. A secure aggregator must then collect data from all these pods, combine these into aggregate statistics, and write this to a new pod accessible to the caretaker.

\newpage

\section{A generic data aggregation architecture in Solid}
\label{sec:generic-agg-arch}
A middleware for aggregation in Solid will consist of a number of components, each with their own responsibility. Figure \ref{fig:reference-architecture} illustrates a reference architecture of a data aggregation system in Solid. 

\begin{figure}[h]
    \centering
    \includegraphics[width=1.0\textwidth]{images/architecture/Reference-Architecture-Aggregator.pdf}
    \caption{Reference architecture for data aggregation in Solid}
    \label{fig:reference-architecture}
\end{figure}

\noindent This reference architecture highlights a number of challenges that make aggregation insecure, unscalable or privacy-invasive. Concretely, the following issues are remarked.

\begin{enumerate}
    \item IoT Sensors and Activity Trackers are embedded devices that very regularly write data to a Solid Pod. As these are resource-constrained devices, those write operations must be very efficient. As section \ref{sec:dpop} explains, the currently used mechanism uses public key cryptography, which is very computationally expensive.
    \item The aggregator component (illustrated in red on the figure) reads complete resources directly from the user's Pod. Privacy-wise, this is unsafe and unnecessary, as not all data present in the resource may be necessary for performing the aggregation. Some method of restricting the data exposed to the aggregator must be devised.
    \item The aggregator component can consist of multiple worker nodes in different networks. Similarly, services can require data from a Solid Pod but also depend on other services that also need access to this data. This is currently not supported in Solid, and existing flows (such as OAuth On-Behalf-Of) impose bottlenecks on the token endpoint. 
    \item Group Pods holding data of users from different Pod providers are not supported as of yet. This comes with challenges in the authentication domain.
\end{enumerate}

\noindent All of these problems are situated (at least partially) in the domain of the Solid server, which provides the user's pods. A number of improvements here could help alleviate the problems listed above. The problems listed above can be divided in two categories: resource management and authentication management. To solve these problems, a solution will need to consist of two parts: one focused on rendering exposed resources more private, the other focused on improving the authentication mechanism. In the next section, some requirements for the needed solution will be derived based on the problems that have been illustrated in this section.

\section{Requirements}
\label{sec:requirements}
This section introduces a number of requirements, based on the use cases and architecture presented in the previous sections. First, functional requirements with a major impact on the architecture are discussed. The second subsection discusses a number of non-functional requirements. These requirements will help designing an effective solution, and their realization will also be evaluated later.

\subsection{Functional requirements}
\noindent \textbf{Flexibility for the supported data scheme} Solid builds further upon the principles of Linked Data, and Solid servers can store nearly any type of data. As \middleware{} aims to be a general solution, it should work with any type of structured data, i.e., it must provide a way to anonymize \textit{any} type of textually represented structured data. Thus, flexibility for the supported data scheme is an essential part of \middleware{}. Concretely, \middleware{} should not hard-code any supported data schemes nor which \gls{PETs} should be applied to them: these should be able to be plugged in flexibly and selected automatically, to ensure that any data scheme can be supported. This provides a technical challenge, as a sufficient level of abstraction must be developed to be able to support any data scheme, while ensuring that the leakage requirements are not violated. \\

\noindent \textbf{Automatically select \gls{PETs}} Different applications may be trusted differently, just as different data schemes may contain more or less sensitive data elements. An important aspect to solve is thus being able to distinguish different \textit{privacy levels}: required levels of anonymization. \middleware{} must therefore not only be able to adapt to different data schemes, but must also support different privacy levels for every data scheme. The selected PET is highly dependent on the data scheme and required level of privacy, therefore \middleware{} must automatically select \gls{PETs} based on the input data scheme and requested privacy level. This requested privacy level must be automatically determined by \middleware{} based on the context of the request: what data type is requested, and by which application. \\

\noindent \textbf{Extensibility} The Solid project is still very much a work in progress. This implies that the specification will likely be modified many times in the future, and additional features will be added. If \middleware{} wants to successfully keep interacting with Solid servers, the architecture should be designed such that it can easily be extended in order to support newly released features and modifications. \\

\noindent \textbf{Decentralized access token delegation} As aggregators may consists of multiple worker nodes, or they may be dependent of other services, the middleware solution must also enable a mechanism for decentralized access token delegation. In other words, an application must be able to delegate its access token without having to request a new token at the token endpoint. The user should explicitly give access for this.

\subsection{Non-functional requirements}
\textbf{Intuitive to use} Solid aims to become the de facto standard for web applications in the future. Consequently, it must be intuitive for non-technical users to use this technology. There can be no technical jargon, and it must be incredibly easy to set-up. Since \middleware{} aims to follow this philosophy, the proposed solution must be intuitive to use for non-technical users.  Concretely, this means that it should be opaque to the users which concrete PET is applied for which use case. On that account, a number of \textit{privacy levels} should be created and presented to the user in a simple manner: a higher privacy level means more data protection but less utility. The user will then be able to select between a number of privacy levels, without needing to understand the technical details behind the scenes. Examples of concrete transformations can be given, to make the effects of selecting a certain level more comprehensible. \\

\noindent \textbf{Testability} Since \middleware{} operates so dynamically, there is a lot of opportunity for bugs to sneak in that will go unnoticed. Some bugs may only appear under the presence of a specific combination of transformations in the configuration files. Therefore, testability is an important aspect. Increased testability gives stronger guarantees of the correct behavior of \middleware{}, which is essential in the context of a privacy-enhancing technology. \\

\noindent \textbf{Minimal overhead} Since \middleware{} operates as a middleware, it will be called on every resource request. The overhead associated with this should be minimized, such that the used Solid server does not become unusable due to running the middleware.

\section{Adversary model}
\label{sec:attacker-model}
The attacker model considered in the design and development of \middleware{} is an honest-but-curious attacker, i.e., an attacker that follows the protocol but will try to read as much information as possible within this confinement. This model is considered since the middleware handles the case of untrusted aggregators and applications. Untrusted in this case means that the user wishes to use the aggregator or application, but wants to minimize the amount of data exposed to it. Thus, the adversary is a Solid application to which the user grants access. Concretely, this means that the application can access any data to which the \gls{ACL}s give it access, and in the modes determined herein. However, there is no fine-grained privacy control, i.e., on the level of a single resource. 

It is important to note the goal of \middleware{} in terms of disclosure prevention here. As will be explained in section \ref{sec:statistical-privacy}, many \gls{PETs} try to prevent identity disclosure (in datasets containing data from many different users). However, since every Solid pod is linked to a WebID (containing data of a single user), the main goal of this middleware is to prevent attribute disclosure. Accordingly, the attackers considered try to gain knowledge of \textit{data attributes}. Handling identity disclosure is also an important aspect of a privacy-aware software system, but in the case of data aggregators for Solid, this must be handled by the aggregators themselves.

\section{System overview}
To solve the problems and fulfill the requirements stipulated above, \middleware{} consists of two main components. The first component is called \textit{privacy filters} and focuses on improving data privacy for data exposed to aggregators (or other applications). The second component is an implementation of macaroons in Solid, i.e. a new access token mechanism. This mechanism supports decentralized delegation, group vaults, and is computationally very efficient (to support data writes by IoT sensors, ...).

Privacy filters are a mechanism to dynamically modify requested resources to strip away or modify sensitive data attributes. When a resource is requested for aggregation, it often contains a lot of unnecessary and private information. Privacy filters are implemented as a middleware in the Solid server, and dynamically strip away certain attributes of the resource. To realize this, a number of configuration files are loaded on to the server, or the user's pod. These configuration files describe what \textit{privacy tactics} ought to be applied to a certain data scheme, for a given \textit{privacy level}. Privacy levels are a granular way to select how much to trust a certain application or aggregator (i.e., how private should the resource be rendered). The next chapter studies a number of \gls{PETs} to determine which are usable in \middleware{}. Privacy filters are then discussed in detail in chapter \ref{cha:privacy-filters}. 

To alleviate the authentication and authorization problems that come with data aggregation in Solid, this dissertation further investigates the use of macaroons in Solid. Some background information on macaroons has been given in section \ref{sec:macaroons}. Macaroons support decentralized delegation out-of-the-box, and also support \textit{caveats} (requirements which must be fulfilled for a token to be valid), as well as third-party attestations. These properties can improve aggregation security, for example by embedding the required privacy level inside the access token, as well as providing a natural mechanism for realizing group vault using the third-party attestation. Furthermore, macaroons allow the access tokens to be restricted to a single resource on the token side. Lastly, macaroons use \acrfull{HMAC} instead of public-key cryptography. This creates performance improvements that enable IoT devices or activity trackers to more efficiently write to Solid pods. The integration of macaroons in Solid is discussed in detail in chapter \ref{cha:macaroons-solid}.

The combination of these two components can enable better and more secure data aggregation in Solid. Figure \ref{fig:aggregation-flow} illustrates more concretely the flow of data aggregation using these two new mechanisms.

\begin{sidewaysfigure}
    \centering
    \includegraphics[width=1.0\textwidth]{images/architecture/InteractionDiagram-Aggregation-flow.pdf}
    \caption{Aggregation flow}
    \label{fig:aggregation-flow}
\end{sidewaysfigure}
