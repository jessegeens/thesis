\chapter{Evaluation}
\label{cha:evaluation}
\section{Architectural limitations}
\todo[inline]{Right now, just some notes}
\begin{itemize}
    \item High performance overhead compared to a solution integrated into the Solid Community Server
    \item Does not support writing back/updating resources, as this would require to keep a mapping when pseudonymization is applied for example -> would need to drastically extend the architecture
    \item Currently, only protects privacy on the basis of a single resource/dataset. Very often, de-anonymization happens because multiple datasets are linked together; protection here is more limited since \middleware{} transforms resources on a one-by-one basis without looking at the total. Of course, not all is lost, because often direct identifiers are removed if the correct privacy level is selected, making it much harder to link datasets together.
    
\end{itemize}
\section{Evaluation of the performance}
\subsection{Experiments set-up}
\subsection{Results}
\section{Evaluation of data leakage}