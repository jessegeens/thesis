\begin{otherlanguage}{dutch}
{\small Solid is een W3C specificatie die tracht het internet te decentraliseren met behulp van datakluizen. Datakluizen zijn gedecentraliseerde opslagplaatsen waar gegevens voor verschillende toepassingen worden bijgehouden. Een uitdagingen hierbij is veilige en beschermde gegevensaggregatie over verschillende datakluizen heen. Dergelijke aggregaties brengen privacyrisico's met zich mee, net als schaalbaarheidsproblemen. Deze thesis introduceert een middleware die tracht deze problemen gedeeltelijk (op het niveau van de server) op te lossen, dankzij het introduceren van privacy filters en een nieuw mechanisme voor toegangstokens dat gedecentraliseerde tokenafvaardiging ondersteunt.
Privacy filters staan toe om meer granulariteit te bereiken in de privacy van gedeelde bronnen. De middleware selecteer hierbij automatisch transformaties die uitgevoerd worden wanneer een bron verzocht wordt, op basis van een aantal contextuele parameters. Om vervolgens een tokenmechanisme te bekomen dat gedecentraliseerde tokenafvaardiging ondersteunt onderzoekt deze thesis het gebruik van macaroons binnen Solid. Macaroons ondersteunen niet alleen gedecentraliseerde tokenafvaardiging, maar zijn ook effici{\"e}nter om te genereren en verifi{\"e}n. Bovendien laten ze attesten van derde partijen toe, wat een nuttige eigenschap is voor datakluizen die gedeeld worden door een groep gebruikers.
De middleware die deze thesis voorstelt werd ge{\"e}valueerd op grond van drie gebruiksscenario's. Voor privacy filters hebben experimenten aangetoond dat de overhead van de middleware toelaatbaar is voor kleinere bronnen met een overhead van ongeveer 50\%. Echter, grotere bronnen (rond de 3MB) hebben een overhead die een vijfvoud wordt van de oorspronkelijke duur van het verzoek. 
Prestatietests die werden uitgevoerd op het genereren en verifi{\"e}ren van macaroons hebben aangetoond dat de doorvoer hiervan bij macaroons respectievelijk zeven en elf maal groter is dan het DPoP (ES256) systeem. Ten slotte zijn er ook theoretische winsten voor gedecentraliseerde tokenafvaardiging.}
\end{otherlanguage}
\\