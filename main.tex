\documentclass[master=cws,masteroption=gs,english]{kulemt}
\setup{title={A middleware for secure and protected data aggregation},
  author={Jesse Geens},
  promotor={Prof.\,dr.\,ir.\ W. Joosen \and dr. B. Lagaisse \and Prof. dr. V. Naessens},
  assessor={Ir.\,W. Eetveel\and W. Eetrest},
  assistant={E. H. Beni \and A. Rafique}}
% De volgende \setup mag verwijderd worden als geen fiche gewenst is.
\setup{filingcard,
  translatedtitle={Een middleware voor veilige en beveiligde dataverzameling},
  udc=621.3,
  shortabstract={Hier komt een heel bondig abstract van hooguit 500
    woorden. \LaTeX\ commando's mogen hier gebruikt worden. Blanco lijnen
    (of het commando \texttt{\string\pa r}) zijn wel niet toegelaten!
    \endgraf \lipsum[2]}}
% Verwijder de "%" op de volgende lijn als je de kaft wil afdrukken
%\setup{coverpageonly}
% Verwijder de "%" op de volgende lijn als je enkel de eerste pagina's wil
% afdrukken en de rest bv. via Word aanmaken.
%\setup{frontpagesonly}

% Kies de fonts voor de gewone tekst, bv. Latin Modern
\setup{font=lm}

% Hier kun je dan nog andere pakketten laden of eigen definities voorzien

% Tenslotte wordt hyperref gebruikt voor pdf bestanden.
% Dit mag verwijderd worden voor de af te drukken versie.
\usepackage[pdfusetitle,colorlinks,plainpages=false]{hyperref}

%%%%%%%
% Om wat tekst te genereren wordt hier het lipsum pakket gebruikt.
% Bij een echte masterproef heb je dit natuurlijk nooit nodig!
\IfFileExists{lipsum.sty}%
 {\usepackage{lipsum}\setlipsumdefault{11-13}}%
 {\newcommand{\lipsum}[1][11-13]{\par Hier komt wat tekst: lipsum ##1.\par}}
%%%%%%%

%\includeonly{chap-n}
\begin{document}

\begin{preface}
  I would like to thank everybody who kept me busy the last year,
  especially my promoter and my assistants. I would also like to thank the
  jury for reading the text. My sincere gratitude also goes to my wive and
  the rest of my family.
\end{preface}

\tableofcontents*

\begin{abstract}
  The \texttt{abstract} environment contains a more extensive overview of
  the work. But it should be limited to one page.

  \lipsum[1]
\end{abstract}

\begin{abstract*}
  In dit \texttt{abstract} environment wordt een al dan niet uitgebreide
  Nederlandse samenvatting van het werk gegeven.
  Wanneer de tekst voor een Nederlandstalige master in het Engels wordt
  geschreven, wordt hier normaal een uitgebreide samenvatting verwacht,
  bijvoorbeeld een tiental bladzijden. 

  \lipsum[1]
\end{abstract*}

% Een lijst van figuren en tabellen is optioneel
%\listoffigures
%\listoftables
% Bij een beperkt aantal figuren en tabellen gebruik je liever het volgende:
\listoffiguresandtables
% De lijst van symbolen is eveneens optioneel.
% Deze lijst moet wel manueel aangemaakt worden, bv. als volgt:
\chapter{List of Abbreviations and Symbols}
\section*{Abbreviations}
\begin{flushleft}
  \renewcommand{\arraystretch}{1.1}
  \begin{tabularx}{\textwidth}{@{}p{12mm}X@{}}
    LoG   & Laplacian-of-Gaussian \\
    MSE   & Mean Square error \\
    PSNR  & Peak Signal-to-Noise ratio \\
  \end{tabularx}
\end{flushleft}
\section*{Symbols}
\begin{flushleft}
  \renewcommand{\arraystretch}{1.1}
  \begin{tabularx}{\textwidth}{@{}p{12mm}X@{}}
    42    & ``The Answer to the Ultimate Question of Life, the Universe,
            and Everything'' according to \cite{h2g2} \\
    $c$   & Speed of light \\
    $E$   & Energy \\
    $m$   & Mass \\
    $\pi$ & The number pi \\
  \end{tabularx}
\end{flushleft}

% Nu begint de eigenlijke tekst
\mainmatter

\include{intro}
\include{chap-1}
\include{chap-2}
% ... en zo verder tot
\include{chap-n}
\include{conclusion}

% Indien er bijlagen zijn:
\appendixpage*          % indien gewenst
\appendix
\include{app-A}
% ... en zo verder tot
\include{app-n}

\backmatter
% Na de bijlagen plaatst men nog de bibliografie.
% Je kan de  standaard "abbrv" bibliografiestijl vervangen door een andere.
\bibliographystyle{abbrv}
\bibliography{references}

\end{document}

%%% Local Variables: 
%%% mode: latex
%%% TeX-master: t
%%% End: 
